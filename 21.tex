\newpage
\chapter{Сериализация объектов}

\section{Концепция и механика сериализации объектов}

\textbf{Сериализация объектов} - это процесс сохранения состояния объектов в виде последовательности байтов. По сути каждый объект (с некоторыми оговорками) можно упаковать в некоторую последовательность файлов, которую затем при помощи потоков и других систем можно передать в другое место с экономией времени и памяти. Для того чтобы можно было управлять данными процессом, используется специальный механизм Java Serialization API.

Для сериализации класса объекта необходимо, чтобы он реализовывал особый интерфейс \verb|java.io.Serializable|. Сам по себе интерфейс не содержит никаких методов для обязательной реализации, он служит маркером для системы, что используемый объект может быть сериализован.

Пример класса, реализующий механизм сериализации:

\begin{lstlisting}
    import java.io.Serializable;
    
    class TestSerial implements Serializable {
      public byte version = 100;
      public byte count = 0;
    }
\end{lstlisting}

Теперь, когда класс может быть сериализован, попробуем записать его в файл \verb|temp.out|:

\begin{lstlisting}
    public static void main(String args[]) throws IOException {
      FileOutputStream fos = new FileOutputStream("temp.out");
      ObjectOutputStream oos = new ObjectOutputStream(fos);
      TestSerial ts = new TestSerial();
      oos.writeObject(ts);
      oos.flush();
      oos.close();
    }
\end{lstlisting}

Ключевым звеном здесь является объект класса \verb|ObjectOutputStream|, который как раз позволяет записать байтовые данные в поток. Для работы над классом можно использовать следующие методы:

\begin{enumerate}
    \item \verb|void close()| — закрытие потока;
    \item \verb|void flush()| — очистка потока и перенос его данных в поток выхода;
    \item \verb|void write(byte[] buf)| — запись массива байтов;
    \item Методы типа \verb|void writeT(T item)|, позволяющие записать значение соответствующего типа, будь-то \verb|int|, \verb|boolean| и пр.;
    \item \verb|void writeObject(Object obj)| — запись самого объекта в поток.
\end{enumerate}

\section{Десериализация}

\textbf{Десериализация} — это обратный сериализации процесс, который позволяет последовательности байтов вновь стать объектом. Для этого в коде Java используется класс \verb|ObjectInputStream|, соответственно извлекающий ранее сериализованные данные из потока.

Возьмём код выше, в котором мы сериализовали объект класса \verb|TestSerial|, и теперь мы десериализуем его:

\begin{lstlisting}
    public static void main(String args[]) throws IOException {
      FileInputStream fis = new FileInputStream("temp.out");
      ObjectInputStream oin = new ObjectInputStream(fis);
      TestSerial ts = (TestSerial) oin.readObject();
      System.out.println("version="+ts.version);
      System.out.println("count="+ts.count);
    }
\end{lstlisting}

При помощи классов \verb|FileInputSteam| и \verb|ObjectInputStream| мы получаем байты информации о классе а затем преобразуем их обратно в объект при помощи метода \verb|readObject()|, возвращающего объект. Приведя его к нужному типу, в консоли вывода мы получим следующую строку:

\begin{lstlisting}
    version=100
    count=0
\end{lstlisting}

Кроме указанного выше метода класс поддерживает также и другие методы:

\begin{enumerate}
    \item \verb|void close()| — закрытие потока;
    \item \verb|int skipBytes(int len)| — пропуск \verb|len| последующих байтов при чтении;
    \item \verb|int available()| — количество оставшихся байтов для чтения;
    \item \verb|int read()| и производные от него — считывает значение одного байта или соответствующего сигнатуре функции объекта.
\end{enumerate}

\section{Классы с запретом на сериализацию и работа с ними}

Несмотря на то, что сериализация предполагает удобный формат передачи данных об объектах класса, иногда она не должна применяться. Так, например, данный процесс не применяется для тех полей, которые:

\begin{enumerate}
    \item вычисляются программно (т.е. на основе другой информации, по ходу выполнения программы);
    \item являются приватными (пароли, уникальные данные и пр.);
    \item не обладают интерфейсом \verb|Serializable| (например, поля классов \verb|Thread|, \verb|OutputStream| и его подклассы, \verb|Socket|;
    \item имеют информацию о состоянии объекта.
\end{enumerate}

Для того чтобы обеспечить отсутствие данных полей с сериализированной последовательности байтов, используется специальное ключевое слово \verb|\textbf{transient}| (технически само поле будет присутствовать, однако его значение всегда будет равно \verb|null|). Так, взяв код определения класса \verb|TestSerial| выше, мы можем запретить сериализацию поля \verb|version|:

\begin{lstlisting}
    import java.io.Serializable;
    
    class TestSerial implements Serializable {
      public transient byte version = 100; // <--
      public byte count = 0;
    }
\end{lstlisting}

Другой способ предполагает полный запрет за сериализацию всего класса вообще — для этого достаточно перегрузить основные методы чтения/записи в поток сериализации:

\begin{lstlisting}
private void writeObject(ObjectOutputStream out) throws IOException

private void readObject(ObjectInputStream in) throws IOException
\end{lstlisting}

При помощи данного кода можно удостовериться, что класс нельзя будет сериализовать подобными функциями — при попытке их вызова выбросится исключение.

\section{Альтернатива сериализации — экстернализация}

Несмотря на то, что сериализация успешно справляется со своими задачами, ей присущи следующие недостатки:

\begin{enumerate}
    \item Низкая производительность за счёт вызова большого количества служебной информации;
    \item Инструментарий ограничения сериализации ограничен лишь запретом отдельных полей через \verb|transient| или класса через перегрузку;
    \item Иногда приходится мучаться со случаями получения "обнулённых" данных — при условии, если пароль и другую конфиденциальную информацию всё же нужно передавать внутренним сервисам.
\end{enumerate}

Для решения данной проблемы имеется альтернатива в виде \textbf{экстернализации} и соответствующего интерфейса \verb|Externalizable|. По сути это та же сериализация, только теперь программист может сам определять правила, по которым данные будут вводиться или выводиться через поток сериализации. В этом случае решаются вышеуказанные проблемы:

\begin{enumerate}
    \item Можно увеличить производительность, вызывая только необходимую информацию;
    \item Инструментарий ограничений сериализации теперь польностью отдан программисту;
    \item Вопрос с паролями и пр. может быть решён за счёт внутренней шифровки чувствительных данных каким-либо алгоритмом шифрования.
\end{enumerate}

Интерфейс \verb|Externalizable| расширяет родительский интерфейс \verb|Serializable| и приводит следующие методы, обязательные к переопределению:

\begin{enumerate}
    \item \verb|void writeExternal(ObjectOutput out)| — запись в поток экстернализации;
    \item \verb|void writeExternal(ObjectInput in)| — чтение из потока экстернализации.
\end{enumerate}

Пример класса, подвергнутого экстернализации, будет выглядеть примерно так:

\begin{lstlisting}
    import java.io.Externalizable;
    import java.io.IOException;
    import java.io.ObjectInput;
    import java.io.ObjectOutput;
    import java.util.Base64;
    
    public class UserInfo implements Externalizable {
    
       private String firstName;
       private String lastName;
       private String password;
    
       private static final long serialVersionUID = 1L;
    
       public UserInfo() {
       }
    
       public UserInfo(String firstName, String lastName, String password) {
           this.firstName = firstName;
           this.lastName = lastName;
           this.password = password;
       }
    
       @Override
       public void writeExternal(ObjectOutput out) throws IOException {
           out.writeObject(this.getFirstName());
           out.writeObject(this.getLastName());
           out.writeObject(this.encryptString(this.getpassword()));
       }
    
       @Override
       public void readExternal(ObjectInput in) throws IOException, ClassNotFoundException {
           firstName = (String) in.readObject();
           lastName = (String) in.readObject();
           password = this.decryptString((String) in.readObject());
       }
    
       private String encryptString(String data) {
           return Base64.getEncoder().encodeToString(data.getBytes());
       }
    
       private String decryptString(String data) {
           return String(Base64.getDecoder().decode(data));
       }
    
       public String getFirstName() {
           return firstName;
       }
    
       public String getLastName() {
           return lastName;
       }
    
       public String getPassword() {
           return password;
       }
    }
\end{lstlisting}

Стоит также отметить, что \textbf{любой класс, имплементирующий интерфейс \texttt{Externalizable}, обязан иметь конструктор по умолчанию} — связано это с особым порядком выполнения десериализации сперва через создание объекта по умолчанию, а затем уже вызова оттуда соответствующих методов.

\label{pages_total}